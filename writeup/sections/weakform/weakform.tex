\documentclass[../../main.tex]{subfiles}

\begin{document}

\section{Constructing the Weak Formulation}

Since FEniCS will be doign the heavy lifting for the matrix construction
and equation solving, the weak formulation will be optimized to fit 
this framework.  
FEniCS is designed around elliptic-like problems,
so the problem will be discretized in time first, 
then each update will be computed by solving an elliptic problem.

The time discretization applied to the Schr\"odinger equation gives
\[
		i \frac{\Psi^{n+1}(\vec x) - \Psi^n(\vec x)}{\Delta t} = 
		\frac{1}{2} \left[-\nabla^2 \Psi^{n+1} - 
		\nabla^2 \Psi^n + \nu V \Psi^{n+1} + \nu V \Psi^n\right]
.\] 
Where $t^n = n \Delta t$ and 
$\Psi^n(\vec x) = \Psi(\vec x, t^n)$.  
Crank-Nicolson can be shown to be second order in time.

Rearranging this above expression gives
\[
		i \Psi^{n+1} + \frac{\Delta t}{2} \nabla^2 \Psi^{n+1} - 
		\frac{\nu \Delta t}{2} V \Psi^{n+1} = 
		i \Psi^n - \frac{\Delta t}{2} \nabla^2 \Psi^n + 
		\frac{\nu \Delta t}{2} V \Psi^n
.\] 
At each time step, we can assume $\Psi^n$ is known, 
with $\Psi^0 \equiv 0$ by the initial condition. 
Therefore this is now an elliptic problem for the $n+1$'st time step.

Next, for a quantum mechanical problem: the wavefunction
must be continuously differentiable, which ensures 
that position and momentum are well defined everywhere.
To guarentee this in $\R^2$ we require 
$\Psi \in H^4(\Omega)$.  

To build the weak form, let 
\[
		\mathcal V = \{u \in H^4(\Omega) : 
		\left.u\right|_{\Gamma_D \cup \Gamma_\mathrm{in}} = 0\}
.\] 
This will be our test function space.
Let $\Phi \in \mathcal V$.  Then the weak formulation 
comes from 3 terms:
\[
  \int_\Omega \Phi \Psi \, dx, \quad 
  \int_\Omega \Phi \nabla^2 \Psi \, dx, \quad 
  \int_\Omega \Phi V \Psi \, dx
.\] 
Here we are considering the real and imaginary components of $\Psi$ separately.
In sticking with the quantum mechanical theme, 
I will be using bra-ket notation.
That is 
$\Psi$ will be known as $\ket{\Psi}$.  
The bar and angle denote that $\Psi$ is an element of the 
Hilbert space $\mathcal V$. 
Next I will denote the linear functional:
\[
		\bra{\Phi} \cdot = \int_\Omega \Phi \cdot \, dx
.\]
That is $\bra{\Phi}$ is the linear functional 
built by the $L^2$ inner product with $\ket{\Phi}$.  
Finally an inner product with an operator $\mathcal O$ will be denoted:
\[
		\bra{\Phi} \mathcal O \ket{\Psi} = 
		\int_\Omega \Phi \mathcal O(\Psi) \, dx
.\] 
The first and second components of the variational form 
are not interesting.  
Instead what is interesting is the middle term.
\[
  \begin{split}
		  \int_\Omega \Phi \nabla^2 \Psi \, dx 
		  & = \bra{\Phi}\nabla^2 \ket{\Psi} \\
		  & = \int_{\partial \Omega} \Phi \pder{\Psi}{n} \, d\sigma - 
		  \int_\Omega \nabla \Phi \cdot \nabla \Psi \, dx \\
		  & = \int_{\Gamma_\mathrm{out}} \Phi \pder{\Psi}{n} \, d\sigma - 
		  \bra{\nabla \Phi}\ket{\nabla \Phi}
  \end{split}
.\] 
Due to the boundary condition on $\Gamma_\mathrm{out}$, we know
\[
		\pder{\Psi}{t} = i \pder{\Psi}{n}
.\] 
Using the Crank-Nicolson time discretization on the boundary
changes this to
\[
		\frac{\Psi^{n+1} - \Psi^n}{\Delta t} = 
		\frac{i}{2} \left(\pder{\Psi^{n+1}}{n} + \pder{\Psi^n}{n}\right)
.\] 
Therefore
\[
		\int_{\Gamma_\mathrm{out}} \Phi \Psi^{n+1} \, d\sigma - 
		\frac{i \Delta t}{2} 
		\int_{\Gamma_\mathrm{out}} \Phi \pder{\Psi^{n+1}}{n} \, d\sigma = 
		\int_{\Gamma_\mathrm{out}} \Phi \Psi^n \, d\sigma + 
		\frac{i \Delta t}{2} 
		\int_{\Gamma_\mathrm{out}} \Phi \pder{\Psi^n}{n} \, d\sigma
.\] 
Giving
\[
		\int_{\Gamma_\mathrm{out}} \Phi \pder{\Psi^{n+1}}{n} \, d\sigma = 
		\frac{2i}{\Delta t} \left(
				-\int_{\Gamma_\mathrm{out}} \Phi \Psi^{n+1} \, d\sigma + 
				\int_{\Gamma_\mathrm{out}} \Phi \Psi^n \, d\sigma + 
				\frac{i \Delta t}{2} 
				\int_{\Gamma_\mathrm{out}} \Phi \pder{\Psi^n}{n} \, d\sigma
		\right)
.\] 
Therefore the weak form of the PDE reads
\begin{equation}
		\label{eq:weak-form-mark-1}
		\begin{split}
				i \bra{\Phi}\ket{\Psi^{n+1}} 
				& - 
				\frac{\Delta t}{2} \bra{\nabla \Phi}\ket{\nabla \Psi^{n+1}} -
				\frac{\nu \Delta t}{2} \bra{\Phi} V \ket{\Psi^{n+1}} -
				i \int_{\Gamma_\mathrm{out}} \Phi \Psi^{n+1} \, d\sigma = \\
				& i \bra{\Phi}\ket{\Psi^n} + 
				\frac{\Delta t}{2} \bra{\nabla \Phi}\ket{\nabla \Psi^n} + 
				\frac{\nu \Delta t}{2} \bra{\Phi} V \ket{\Psi^n} - 
				i \int_{\Gamma_\mathrm{out}} \Phi \Psi^n \, d\sigma 
		\end{split}
\end{equation}

FEniCS uses UFL - unified form assembly language, which 
is able to take a (nearly verbatim mathematically formatted) 
variational form, and convert this into a matrix system and solve 
automatically.  
The only adjustment that needs to be made is to the boundary integrals. 
To enforce the integration take place only on 
$\Gamma_{\mathrm{out}}$, we use 
\[
		\int_{\Gamma_\mathrm{out}} \Phi \Psi \, d\sigma = 
		\int_{\partial \Omega} \Phi \mathbbm 1(\vec x \in \Gamma_\mathrm{out})
		\Psi \, d\sigma
.\] 


\end{document}
