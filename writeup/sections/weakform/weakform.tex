\documentclass[../../main.tex]{subfiles}

\begin{document}

\section{Constructing the Weak Formulation}

Since FEniCS will be doign the heavy lifting for the matrix construction
and equation solving, the weak formulation will be optimized to fit 
this framework.  
FEniCS does not support complex numbers.  
To deal with this, we will use FEniCS to build the matrix formulation 
of the problem, then deal with an ODE solver which can handle complex
numbers. 

In sticking with the quantum mechanical theme, 
I will be using bra-ket notation (little bit late for Halloween, but okay).
That is 
$\Psi$ will be known as $\ket{\Psi}$.  
The term inside the ket: $\ket{\cdot}$ can be anything,
as it is a label.  For now we will leave it as $\ket{\Psi}$.
The bar and angle denote that $\Psi$ is an element of the 
Hilbert space $\mathcal V$. 
Next I will denote the linear functional by a bra:
\[
		\bra{\Phi} \cdot = \int_\Omega \Phi^* \cdot \, dx
.\]
That is $\bra{\Phi}$ is the linear functional 
built by the $L^2$ inner product with $\ket{\Phi}$.
Again, what is inside the $\bra{\times}$ is simply a label,
and for now it will be left as $\bra{\Phi}$
Finally an inner product with an operator $\mathcal O$ will be denoted:
\[
		\bra{\Phi} \mathcal O \ket{\Psi} = 
		\int_\Omega \Phi^* \mathcal O[\Psi] \, dx
.\] 

On the interior of the domain, the Schr\"odinger equation can now
be written in variational form:
\[
		  i \bra{\Phi} \partial_t \ket{\Psi} 
		   =  \bra{\Phi} -\nabla^2 \ket{\Psi} + 
		  \bra{\Phi} V \ket{\Psi}
.\]
At this point, we can integrate the laplacian term by parts:
\[
  \begin{split}
		  \bra{\Phi} -\nabla^2 \ket{\Psi} 
		  & = -\int_\Omega \Phi^* \nabla^2 \Psi \, dx \\
		  & = -\int_{\partial \Omega} \Phi^* \pder{\Psi}{n} \, d\sigma + 
				  \int_\Omega \nabla \Phi^* \cdot \nabla \Psi \, dx \\
		  & = -\int_{\partial \Omega} \Phi^* \pder{\Psi}{n} \, d\sigma + 
		  \bra{\nabla \Phi} \ket{\nabla \Psi}
  \end{split}
.\] 
For $\mathcal V$ we choose the subspace of $H^1(\Omega)$ which is 
zero on both $\Gamma_D$ and $\Gamma_\mathrm{in}$.  
This term then becomes
\[
		\bra{\Phi} -\nabla^2 \ket{\Psi} = 
		-\int_{\Gamma_\mathrm{out}} \Phi^* \pder{\Psi}{n} \, d\sigma 
		+ \bra{\nabla \Phi} \ket{\nabla \Psi}
.\] 

Next, on $\Gamma_\mathrm{out}$, the boundary condition was chosen 
such that 
\[
		\pder{\Psi}{t} = i \pder{\Psi}{n} \implies 
		\int_{\Gamma_\mathrm{out}} \Phi^* \pder{\Psi}{n} \, d\sigma = 
		-i \int_{\Gamma_\mathrm{out}} \Phi^* \pder{\Psi}{t} \, d\sigma
.\] 
Therefore the Schr\"odinger equation now reads
\[
		i \bra{\Phi} \partial_t \ket{\Psi} = 
		\bra{\nabla \Phi} \ket{\nabla \Psi} + 
		\bra{\Phi} V \ket{\Psi} + 
		i \int_{\Gamma_\mathrm{out}} \Phi^* \pder{\Psi}{t} \, d\sigma
.\] 

At this point, we can make a finite dimensional approximation.
FEniCS will generate a triangulation, $\mathcal T_h$ for 
$\Omega$.  
Using standard $C^0-P^1$ elements, 
we get a Lagrange basis for our finite dimensional subspace of 
$\mathcal V$.
It should be noted we are taking $\mathcal V$ to be a real space,
the complex parts of the wave functions will be incorperated into 
the time variation.
Let this basis for $\mathcal V$ be denoted by $\phi_j$ for $j = 0, 1, ..., N$. 
We then make the ansantz
\[
		\Psi(x, t) = \sum_j \alpha_j(t) \phi_j(x)
.\] 
Here $\alpha_j : \R_+ \to \C$ are complex functions, 
and $\phi_j$ are real.
The Schr\"odinger equation reads in variational form
\[
		i \sum_\ell 
		\underbrace{\bra{\phi_k} \ket{\phi_\ell}}_{M_{k\ell}} \alpha_\ell' = 
		\sum_\ell 
		\left(\underbrace{\bra{\nabla \phi_k} \ket{\nabla \phi_\ell}}_{
				S_{k\ell}} + 
		\underbrace{\bra{\phi_k} V \ket{\phi_\ell}}_{
			V_{k\ell}}\right) \alpha_\ell + 
		i \sum_\ell \alpha_\ell' 
		\underbrace{
				\int_{\Gamma_\mathrm{out}} \phi_k \phi_\ell 
		\, d\sigma}_{B_{k\ell}}
.\] 
Where $M$ is the mass matrix, $S$ the stiffness matrix,
$V$ is a ``potential'' matrix, and $B$ is a ``boundary'' matrix.
In matrix form this reads
\[
		i (M - B) \alpha' = (S + V) \alpha
.\]
Here $M, B, S, V$ are all real matrices, and $\alpha$ are complex vectors.

Once the matrices are assembled, time stepping is done using the 
Crank-Nicolson method.  
This reads to solving
\[
		\left[i (M - B) - \frac{\Delta t}{2} (S + V)\right] \alpha^{n+1} = 
		\left[i (M - B) + \frac{\Delta t}{2} (S + V) \right] \alpha^n
.\] 
For $\alpha^{n+1}$ where $\alpha^n = \alpha(n \Delta t)$ is assumed to
be given.
The initial condition: $\Psi \equiv 0$ translates to 
$\alpha^0 = 0$.

It should be noted that to account for the input boundary condition,
we can index the matrices using 
$I_D$ for the nodes on $\Gamma_D$ and 
$I_\mathrm{in}$ for the nodes on $\Gamma_\mathrm{in}$.
Since on $\Gamma_D$, $\Psi = 0$ this does not contribute.  
However we can extract the free indices (those not in $I_D$ or $I_\mathrm{in}$)
and extract those on $I_\mathrm{in}$.  
Then the problem reads as:
\[
  \begin{split}
		  \left[i (M - B) - \frac{\Delta t}{2} (S + V)\right]_\mathrm{free} 
		  \alpha^{n+1}_\mathrm{free}
		  & = \left[i(M - B) + \frac{\Delta t}{2} (S + V)\right]_\mathrm{
		  free + in} \alpha^n_\mathrm{free + in} \\
		  & - \left[i(M - B) + \frac{\Delta t}{2} (S + V)\right]_\mathrm{in} 
		  f\left(x\in \Gamma_\mathrm{in}, t^{n+1}\right)
  \end{split}
.\] 
\end{document}
