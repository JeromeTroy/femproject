\documentclass[../../main.tex]{subfiles}

\begin{document}

\section{Constructing the Weak Formulation}

To construct the weak formulation of the nondimensionalized 
Schr\"odinger equation, consider a test function 
$\phi(\vec x)$ which is sufficiently smooth
(this will be clarified later). 
Integrating against the Schr\"odinger equation gives
\[
		i\inner{\phi, \pder{\Psi}{t}}_{L^2} = 
		\inner{\nabla \phi, \nabla \Psi}_{L^2} + 
		\nu \inner{\phi, V \Psi}_{L^2}
.\] 
Where each integral is on all of $\R^2$.  
Importantly, the quantity 
$\Psi^* \Psi$ represents a probability density function (pdf) 
for the particle.  
As a result, since the probability the particle exists somewhere 
in $\R^2$ is 1, $\Psi \in L^2(\R^2) \times C^1(\R_+) $.  
Applying the same approach to the initial condition:
$\Psi_0(\vec x) := \Psi(\vec x, 0)$ gives 
\[
		\inner{\nabla \phi, \nabla \Psi_0}_{L^2} + 
		\nu \inner{\phi, V^{(0)} \Psi_0}_{L^2} = 
		\epsilon \inner{\phi, \Psi_0}
.\] 
For the weak form to exist, 
$\Psi \in H^1(\R^2) \times C^1(\R_+)$.  
However there is more to it than this.  
In quantum mechanics, the operator
\[
		\hat {\vec p} = -i \hbar \nabla
\] 
corresponds to the momentum of a particle;
and the pdf for the momentum is represented by
$\Psi^* \cdot \hat{\vec p}(\Psi)$ \cite{griffiths-quantum}.
Consequentially, requiring the momentum be well defined everywhere
necessitates $\Psi \in C^1(\R^2) \times C^1(\R_+)$.  
To ensure this, we must therefore require 
$\Psi \in H^2(\R^2) \times C^1(\R_+)$.  
Therefore we can now construct the weak form of the problem
\begin{gather}
		i \inner{\phi, \pder{\Psi}{t}}_{L^2(\R^2)} = 
		\inner{\nabla \phi, \nabla \Psi}_{L^2(\R^2)} + 
		\nu \inner{\phi, V\Psi}_{L^2(\R^2)} \\
		\inner{\nabla \phi, \nabla \Psi_0}_{L^2(\R^2)} + 
		\nu \inner{\phi, V^{(0)} \Psi_0}_{L^2(\R^2)} = 
		\epsilon_\mathrm{min} \inner{\phi, \Psi_0}_{L^2(\R^2)}, 
		\quad \forall \phi \in H_0^2(\R^2)
\end{gather}

\end{document}
