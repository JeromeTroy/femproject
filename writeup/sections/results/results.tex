\documentclass[../../main.tex]{subfiles}

\begin{document}

\section{Results and Discussion}

We tested the solution scheme using a quasi-uniform mesh
with the minimum cell diameter: $h_\mathrm{min} = 0.1$.
The time stepping used a maximum time of 5 with $5000$ time steps.  
Below are several figures describing the time evolution of the problem.
Each figure has the probability density function (pdf) plotted,
where the pdf is given by
\[
		\rho(x, t) = \Psi^*(x, t) \Psi(x, t) = 
		\sum_j |\alpha_j(t)|^2 \phi_j^2(x)
.\] 
Which is pointwise multiplication.  
It should be noted that what is computed is not a pdf, since its 
integral may not be 1.  
However taking ratios of integrals will give the relative probabilities 
of the particle existing in specified regions 
\cite{wasserman-statistics}. 
Further, the pdf is plotted using a log scale, to better see
regions of low probability.


The results show that the wave carries its momentum through the system,
and initially behaves as if it were a car going around a turn.  
However after time $t = 1.5$, the system equilibrates,
and the wave function no longer appears to exhibit a preferential 
traversal direction.  These figures can be seen in the appendix.

Next consider the transmission coefficient.  
This is given by the output amplitude divided by the input 
amplitude.  Here we compute this via the following ratio
\begin{equation}
		\label{eq:transmission}
		C_\mathrm{trans} = \frac{
		\bra{\Psi} \mathbbm 1_{\Omega_0} \ket{\Psi}}{
		\bra{\Psi} \mathbbm 1_{\Omega_1} \ket{\Psi}} = 
		\left(\int_{\Omega_1} \rho(x, t) \, dx\right)^{-1} 
		\int_{\Omega_0} \rho(x, t) \, dx
.\end{equation}
This quantity is variable with time.  
This can be seen in figure \ref{fig:trans-coef}.  
Here the transmission coefficient can be seen to peak at 
just below 4\%.  
Furthermore it can be seen transmission starts to occur after time 
$t = 1$.  
It should be noted that the eigenvalues of the Schr\"odinger equation
are all purely imaginary, which lie on the boundary 
of the stability region of the time stepping method - Crank-Nicolson.
As a result, it is not surprising to see some high oscillation modes
in the transmission coefficient.


\end{document}
