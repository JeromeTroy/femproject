\documentclass[../../main.tex]{subfiles}

\begin{document}

\section{Figures for Results}

\begin{figure}[h]
		\centering
		\includegraphics[width=0.75\textwidth]{../../wave_entrance.png}
		\caption{Entrace profile of the wave into the input tube,
				taken at $t = 0.2$. 
				The wave has entered the input tube but has yet to enter the 
		ring resonator}
		\label{fig:wave-entrance}
\end{figure}

\begin{figure}[h]
		\centering
		\includegraphics[width=0.75\textwidth]{../../wave_enters_ring.png}
		\caption{At this point, the wave has spilled over into the ring 
				resonator,
				taken at $t = 0.4$. 
				Note that the momentum of the wave (towards the positive
				$y$ direction) causes it to fill the 
				top part of the ring first.
		}
		\label{fig:wave-enters-ring}
\end{figure}

\begin{figure}[h]
		\centering
		\includegraphics[width=0.75\textwidth]{../../wave_completes_ring.png}
		\caption{The wave covers nearly all the ring at this point,
				albeit at a much lower intensity than at the input.
				This occurs at $t = 0.6$.
		}
		\label{fig:wave-completes-ring}
\end{figure}

\begin{figure}[h]
		\centering
		\includegraphics[width=0.75\textwidth]{../../wave_exits_ring.png}
		\caption{Once the wave fills the ring nearly uniformly,
				the wave beings to spill over into the exit tube.
				This snapshot was taken at time $t = 0.8$.
		}
		\label{fig:wave-exits-ring}
\end{figure}

\begin{figure}[h]
		\centering
		\includegraphics[width=0.75\textwidth]{../../wave_fills_exit_bottom.png}
		\caption{At time $t = 1$ the wave reaches the exit point
				of the output tube.  
				It's initial preference is having flipped 180$^\circ$ 
				and returning to the direction it entered.
		}
		\label{fig:wave-exits-bottom}
\end{figure}

\begin{figure}[h]
		\centering
		\includegraphics[width=0.75\textwidth]{../../wave_fills_exit.png}
		\caption{By time $t = 1.5$ the wave has come to a relatively
				stable state, and has filled the exit tube. 
				It no longer appears to exhude a preference for 
				exiting out the top or bottom.
		}
		\label{fig:wave-exits}
\end{figure}

\begin{figure}[h]
		\centering
		\includegraphics[width=0.75\textwidth]{../../trans_coef.pdf}
		\caption{Time variation of transmission coefficient.  
				It can be seen the tranmission peaks after 
				$t = 3.5$ then dips back down.  
				It can also be seen that no more than 4\% of the wave 
				is ever transmitted through to the output tube.
		}
		\label{fig:trans-coef}
\end{figure}

\end{document}
