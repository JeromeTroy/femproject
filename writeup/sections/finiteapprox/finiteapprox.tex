\documentclass[../../main.tex]{subfiles}

\begin{document}
\section{Finite Dimensional Approximation}

Since FEniCS cannot handle complex numbers, we are considering the 
real and imaginary components of $\Psi$ separately. 
To that end, thus far all variational terms are between real functions.
Next, we suppose
\[
  \Psi = \Psi_R + i \Psi_I
.\] 
Where $\Psi_{R, I}$ are real functions corresponding to the 
real and imaginary components of $\Psi$ respectively.  
Let $H^1(\Omega)$ be the function space considered
(Note however that the potential needs to only live in $L^2$).
Let $\psi_\ell$ be the $\ell$'th basis function in the 
finite dimensional approximation of $H^1$, which is build on top of 
a mesh built in FEniCS.  
FEM demands
\[
		\Psi = \sum_\ell \lambda_\ell(t) \phi_\ell, \quad 
		\lambda_\ell : \R^+ \to \C
.\] 
Here $\lambda_\ell$ are the (complex) scalings of the basis functions.
This takes care of the fact that $\Psi$ is complex. 

Now the bra-ket notation can really shine.  In the ket $\ket{\Psi}$, 
$\Psi$ is just a dummy indicator.  So it can be anything we want.
To that end, let
\[
		\bra{k} \ket{\ell} = \int_\Omega \psi_k \psi_\ell \, dx, \quad
		\bra{D, k} \ket{D, \ell} = 
		\int_\Omega \nabla \psi_k \cdot \nabla \psi_\ell \, dx
.\] 
And so on... This means the mass, stiffness and potential 
matrices are indexed by
\[
		M_{k\ell} = \bra{k} \ket{\ell}, \quad 
		S_{k\ell} = \bra{D, k} \ket{D, \ell}, \quad 
		V_{k\ell} = \bra{k} V \ket{\ell} = 
		\int_\Omega \psi_k V \psi_\ell \, dx
.\] 
We also denote a boundary operator:
\[
		B_{k \ell} = \int_{\Gamma_\mathrm{out}} \psi_k \psi_\ell \, d\sigma
.\] 
Then the finite dimensional variational form reads
\begin{equation}
		\begin{split}
		\label{eq:fd-variational}
		\sum_\ell &\left(i \bra{k}\ket{\ell} - 
				\frac{\Delta t}{2} \bra{D, k}\ket{D, \ell} - 
				\frac{\nu \Delta t}{2} \bra{k} V \ket{\ell} - 
				i B_{k \ell} 
		\right) \lambda_\ell^{n+1} = \\
		& \sum_\ell \left(i \bra{k}\ket{\ell} + 
				\frac{\Delta t}{2} \bra{D, k} \ket{D, \ell} + 
				\frac{\nu \Delta t}{2} \bra{k} V \ket{\ell} - 
				i B_{k \ell}
		\right) \lambda_\ell^n
		\end{split}
\end{equation} 
 
Finally the boundary conditions.
All remaining boundary conditions are Dirichlet.  
Let $I_D$ correspond to the indices of coordinates on $\Gamma_D$ 
and $I_\mathrm{in}$ those on $\Gamma_\mathrm{in}$.  
These indices can be extracted in the above matrices,
and assigned in $\lambda^{n+1}$ after solving.
The remaining indices:
$I_\mathrm{free}$ are the free indices which need to be solved to 
determine the appropriate values.


\end{document}
